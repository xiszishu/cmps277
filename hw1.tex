\documentclass{article}
\usepackage{times}
\usepackage{amssymb}
\usepackage{subfigure}
\renewcommand\thesubsection{\thesection.\alph{subsection}}

\begin{document}

\title{CPMS277 Homework 1}
\author{Xiao Liu\\
  \emph{xiszishu@ucsc.edu}}
\date{\today}
\maketitle

\section{}
The number of distinct relations are:\\
\begin{center}
$2^{\prod\limits_{i=1}^{k}(n_{i}+1)}$
\end{center}
In each tuple, it can choose from all the elements of each attribute or no element which are $n_{i}+1$ choices. So the number of all the tuples is the product of all the possible choices from each attribute. And we can decide whether the tuple is in the relation.\\

\section{}

Let $S=\phi$ and $R\neq\phi$\\
Then $\pi_{A}(R\times S)=\phi\neq\pi_{A}(R)$\\
Therefore  the statement $\pi_{A}(R\times S)=\pi_{A}(R)$ is false\\

\section{}
$R/S=\pi_{A_{1},...,A_{m}}(R)-\pi_{A_{1},...,A_{m}}(\pi_{A_{1},...,A_{m}}(R)\times S-R)$

\section{}
Selection monotone proof:\\
Suppose there is a selection $\sigma_{A}$,and two relations: $R_{1}\subseteq R_{2}$\\
If $\mu$ is a random tuple in $\sigma_{A}(R_{1})$, then $\mu$ is in $R_{2}$.\\
In addition, $\mu$ satisfies condition $A$, thus $\mu$ is in $\sigma_{A}(R_{2})$.\\
Given that $\mu$ is a random tuple, this proof applies to all the tuples in $\sigma_{A}(R_{1})$, statement $\sigma_{A}(R_{1})\subseteq \sigma_{A}(R_{2})$ is tenable.\\
\\
Projection monotone proof:\\
Suppose there is a projection $\pi_{A}$,$A$ is an attribute list has one or multiple attributes,and two relations: $R_{1}\subseteq R_{2}$.\\
If $\mu$ is a random tuple in $\pi_{A}(R_{1})$, since $R_{1}$ has same attributes with $R_{2}$, then $\mu$ shares same attributes with $R_{2}$.\\
In addition, $\mu$ has the same attributes in the list $A$, thus $\mu$ is in $\pi_{A}(R_{2})$.\\
Given that $\mu$ is a random tuple, this proof applies to all the tuples in $\pi_{A}(R_{1})$, statement $\pi_{A}(R_{1})\subseteq \pi_{A}(R_{2})$ is tenable.\\
\\
Join monotone proof:\\
Suppose there is an arbitrary relation $S$, and two relations: $R_{1}\subseteq R_{2}$.\\
If $\mu$ is a random tuple in $R_{1}\ltimes S$, $\mu$ is in $R_{1}$, then $\mu$ is in $R_{2}$.\\
In addition, $\mu$ shares same attributes with $S$, thus $\mu$ is in $R_{2}\ltimes S$.\\
Given that $\mu$ is a random tuple, this proof applies to all the tuples in $R_{1}\ltimes S$, statement $(R_{1}\ltimes S) \subseteq (R_{2}\ltimes S)$ is tenable.\\
This proof could be applied to semijoin and thetajoin easily.\\
\\
Union monotone proof:\\
Suppose there is an arbitrary relation $S$, and two relations: $R_{1}\subseteq R_{2}$.\\
If $\mu$ is a random tuple in $R_{1}\cup S$, then $\mu$ is whether in $R_{1}$ or $S$.\\
1. If $\mu$ is in $R_{1}$, then $\mu$ is in $R_{2}$, thus  $\mu$ is in $R_{2}\cup S$.\\
2. If $\mu$ is in $S$, then $\mu$ is in $R_{2}\cup S$.\\
So in both conditions,  $\mu$ is eventually in $R_{2}\cup S$.\\
Given that $\mu$ is a random tuple, this proof applies to all the tuples in $R_{1}\cup S$, statement $(R_{1}\cup S)\subseteq (R_{2}\cup S)$ is tenable.\\
\\
Difference monotone disproof:\\
Suppose $R_{1}=R_{1}^{*}$ and $R_{2}=R_{2}^{*}$, the domains of these 4 relational schemas is all the alphabets.\\
Thus, we have $R_{1}\subseteq R_{1}^{*}$ and $R_{2}\subseteq R_{2}^{*}$.\\
Let $r_{1}$,$r_{2}$,$r_{1}^{*}$ and $r_{2}^{*}$ be the relations of $R_{1}$,$R_{2}$,$R_{1}^{*}$ and $R_{2}^{*}$ respectively.\\
Suppose
\begin{table}[htbp]
\begin{minipage}{.3\linewidth}
\centering
\begin{tabular}{|c|c|}
\hline 
$r_{1}$ &  A \\
\hline
& a  \\
& b  \\
& c  \\
& d  \\
& e  \\
\hline 
\end{tabular}
\end{minipage}\begin{minipage}{.3\linewidth}  
\centering
\begin{tabular}{|c|c|}
\hline 
$r_{2}$ & A \\
\hline
& a  \\
& b  \\
& c  \\
\hline 
\end{tabular}
\end{minipage}\begin{minipage}{.3\linewidth}
\centering
\begin{tabular}{|c|c|}
\hline
$r_{1}^{*}$ & A \\
\hline
& a  \\
& b  \\
& c  \\
& d  \\
& e  \\
\hline
\end{tabular}
\end{minipage}\begin{minipage}{.3\linewidth}
\centering
\begin{tabular}{|c|c|}
\hline
$r_{2}^{*}$ & A \\
\hline
& a  \\
& b  \\
& c  \\
& d  \\
\hline
\end{tabular}
\end{minipage}

\end{table}
\\
Thus we have $r_{1}\subseteq r_{1}^{*}$ and $r_{2}\subseteq r_{2}^{*}$.\\
Cleanly, $r_{1}-r_{2}\notin r_{1}^{*}-r_{2}^{*}$.\\
Therefore, difference is not monotone.\\
\\
Given the 5 proofs above, difference query is not monotone while SPJU queries are monotone, difference of $r_{1}$ and $r_{2}$ cannot be expressed with the relational algebra operators: select,project,join and union.


\section{}
\subsection{}
$R\ltimes S=\pi_{attr(R)}(R\bowtie S)$
\subsection{}
If $R$ and $S$ share no common attributes, then $R\bowtie S=\phi$\\
Therefore $R\ltimes S=\phi$\\
\subsection{}
No\\
Cartesian product could bring new attributes to a relation.\\
While select, project, difference, semijoin and union could not.Even though rename could change attributes' name, it cannot bring new attribute to the relation either.\\
In another way, we cannot use the replaced operators to express cartesian product, also shows the result is relationally incomplete.\\

\subsection{}
No\\
Even though both difference and intersect could decrease the number of tuples in a relation, but the difference is still irreplaceable. You can only use difference to get the tuples that is unique in multiple relations.\\
Besides, difference is one of the fundamental operator, cannot be replaced by other fundamental or non-fundamental operator.\\

\end{document}
